\documentclass{ndjflart}
%%% HIGHLY RECOMMENDED PACKAGES AND SETTINGS
%\usepackage{pdfsync}  %% if you know what this is use it or not.
\usepackage[T1]{fontenc}
%%%%%%%%%%%%%%%%%%%%%%%%%%%%%%
%% If your tex system is less than 2 years old (in 2012) the following
%% font options are available. If not comment them out.
\usepackage{tgtermes}
%       otherwise use alternative journal fonts
%\renewcommand{\rmdefault}{ptm} % system default Times font
\usepackage{mathptmx}
%%%     additional fonts
\usepackage[scaled=.92]{helvet}
%\setoptfont{enc={T1},fam={pop}} % if You have Optima font, uncomment this line
%%% MATH
\usepackage{amsthm,amsmath,amssymb}
\usepackage{mathrsfs}
%%% BIBLIOGRAPHY
\usepackage[numbers]{natbib}  %% numbers is required.
%%% LINKS
\usepackage[colorlinks,citecolor=blue,urlcolor=blue]{hyperref}  %%check

\usepackage{enumerate}

\artstatus{am} %%% leave this alone!!  That means you, too!!

\startlocaldefs
\DeclareMathOperator{\emb}{emb}
\endlocaldefs

\begin{document}

\section{Open questions}
The notion of quasi-homogeneity introduced in this article does not seem to
appear in the literature.
Studying it in more depth may be a worthwhile direction for further
research since currently our knowledge of it is very limited.
For instance, we know that homogeneity and quasi-homogeneity coincide over
countable structures
but how much they really differ over uncountable structures is a question
that we leave open.

We also leave open the question asked by the referee about a
converse to Theorems 5.5 and 5.20.
Given a chain of structures, does having every formula of
$\mathcal{L}_{\infty \omega}(\mathcal{Q}_{\emb})$
eventually equivalent to some fixed quantifier-free formula in this chain
imply that it eventually consists only of structures that are interpretations
of quasi-homogeneous structures for some fixed interpretation?

\end{document}
